\begin{longtable}{Xr}
	Mina Huh, Yi-Hao Peng, \textbf{Amy Pavel}. ``GenAssist: Making Image Generation Accessible'' \textit{To Appear at UIST 2023} & October 2023 \\
	\\

	Daniel Killough, \textbf{Amy Pavel}. ``Exploring Community-Driven Descriptions for Making Livestreams Accessible'' \textit{To Appear at ASSETS 2023} & October 2023 \\
	\\

	Mina Huh, Saelyne Yang, Yi-Hao Peng, Xiang "Anthony" Chen, Young-Ho Kim, \textbf{Amy Pavel}. ``AVscript: Accessible Video Editing with Audio-Visual Scripts'' \textit{CHI 2023} & April 2023 \\
	\\

	Jeremy Warner, \textbf{Amy Pavel}, Tonya Nguyen, Maneesh Agrawala, Björn Hartmann. ``SlideSpecs: Automatic and Interactive Presentation Feedback Collation'' \textit{IUI 2023} & April 2023 \\
	\\

	Yi-Hao Peng, Jason Wu, Jeffrey P. Bigham, \textbf{Amy Pavel}. ``Diffscriber: Describing Visual Design Changes to Support Mixed-Ability Collaborative Presentation Authoring'' \textit{UIST 2022} & October 2022 \\
	\\

	Xingyu Liu, Ruolin Wang, Dingzeyu Li, Xiang "Anthony" Chen, \textbf{Amy Pavel}. ``CrossA11y: Identifying Video Accessibility Issues via Cross-modal Grounding'' \textit{UIST 2022} & October 2022 \\
	\\

	Yasmine Kotturi, Herman T Johnson, Michael Skirpan, Sarah E Fox, Jeffrey P. Bigham, \textbf{Amy Pavel}. ``Tech Help Desk: Support for Local Entrepreneurs Addressing the Long Tail of Computing Challenges'' \textit{CHI 2022} & April 2022 \\
	\\

	Candace Williams, Lilian de Greef, Ed Harris III, \textbf{Amy Pavel}, Cynthia L. Bennett. ``Toward supporting quality alt text in computing publications'' \textit{W4A 2022} & April 2022 \\
	\\

	Junhan Kong, Dena Sabha, Jeffrey P. Bigham, \textbf{Amy Pavel}, Anhong Guo. ``TutorialLens: authoring Interactive augmented reality tutorials through narration and demonstration'' \textit{SUI 2021} & November 2021 \\
	\\

	Yi-Hao Peng, Jeffrey P. Bigham, \textbf{Amy Pavel}. ``Slidecho: Flexible Non-Visual Exploration of Presentation Videos'' \textit{ASSETS 2021} & October 2021 \\
	\\

	Stephanie Valencia, Michal Luria, \textbf{Amy Pavel}, Jeffrey P. Bigham, Henny Admoni. ``Co-designing Socially Assistive Sidekicks for Motion-based AAC'' \textit{HRI 2021} & March 2021 \\
	\\

	Xingyu Liu, Patrick Carrington, Xiang "Anthony" Chen, \textbf{Amy Pavel}. ``What Makes a Video Non-Visually Accessible?'' \textit{CHI 2021} & May 2021 \\
	\\

	Yi-Hao Peng, JiWoong Jang, Jeffrey P. Bigham, \textbf{Amy Pavel}. ``Say It All: Feedback for Improving Non-Visual Presentation Accessibility'' \textit{CHI 2021} & May 2021 \\
	\\

	Prakhar Gupta, Jeffrey P. Bigham, Yulia Tsvetkov, \textbf{Amy Pavel}. ``Controlling Dialogue Generation with Semantic Exemplars.'' \textit{NAACL 2021} & June 2021 \\
	\\

	\textbf{Amy Pavel}, Gabriel Reyes, Jeffrey P. Bigham. ``Rescribe: Authoring and Automatically Editing Audio Descriptions.'' \textit{UIST 2020} (\textasciitilde22\% acceptance rate, 10 pages) -- Highlighted in Future of CSCW/UIST Plenary, and UIST Keynote. & October 2020 \\
	\\

	Cole Gleason, Stephanie Valencia, Lynn Kirabo, Jason Wu, Anhong Guo, Elizabeth J. Carter, Jeffrey P. Bigham, Cynthia L. Bennett, \textbf{Amy Pavel}. ``Disability and the COVID-19 Pandemic: Using Twitter to Understand Accessibility during Rapid Societal Transition.'' \textit{ASSETS 2020} (28\% acceptance rate, 10 pages) & October 2020 \\
	\\

	Cole Gleason, \textbf{Amy Pavel}, Himalini Gururaj, Kris M. Kitani, Jeffrey P. Bigham. ``Making GIFs Accessible.'' \textit{ASSETS 2020} (28\% acceptance rate, 10 pages) & October 2020 \\
	\\

	Jaylin Herskovitz, Jason Wu, Samuel White, \textbf{Amy Pavel}, Gabriel Reyes, Anhong Guo, Jeffrey P. Bigham. ``Making Mobile Augmented Reality Applications Accessible.'' \textit{ASSETS 2020} (28\% acceptance rate, 10 pages) & October 2020 \\
	\\

	Stephanie Valencia, \textbf{Amy Pavel}, Jared Santa Maria, Seunga (Gloria) Yu, Jeffrey P. Bigham, Henny Admoni. ``Conversational Agency in Augmentative and Alternative Communication.'' \textit{CHI 2020} (24.3\% acceptance rate, 10 pages) -- Best Paper Honorable Mention & May 2020 \\
	\\

	Cole Gleason, \textbf{Amy Pavel}, Emma McCamey, Christina Low, Patrick Carrington, Kris M. Kitani, Jeffrey P. Bigham. ``Twitter A11y: A Browser Extension to Make Twitter Images Accessible.'' \textit{CHI 2020} (24.3\% acceptance rate, 10 pages) -- Best Paper Honorable Mention & May 2020 \\
	\\

	Prakhar Gupta, Shikib Mehri, Tiancheng Zhao, \textbf{Amy Pavel}, Maxine Eskenazi, Jeffrey P. Bigham. ``Investigating Evaluation of Open-Domain Dialogue Systems With Human Generated Multiple References.'' \textit{SIGDIAL 2019} (10 pages) & October 2019 \\
	\\

	Cole Gleason, \textbf{Amy Pavel}, Xingyu Liu, Patrick Carrington, Lydia Chilton, Jeffrey P. Bigham. ``Making Memes Accessible.'' \textit{ASSETS 2019} (26\% acceptance rate, 10 pages) & October 2019 \\
	\\

	Vincent Sitzmann, Ana Serrano, \textbf{Amy Pavel}, Maneesh Agrawala, Diego Gutierrez, Belen Masia, Gordon Wetzstein. ``Saliency in VR: How do people explore virtual environments?'' \textit{IEEE VR 2018} (22.5\% acceptance rate, 9 pages) & March 2018 \\
	\\

	\textbf{Amy Pavel}, Björn Hartmann, Maneesh Agrawala. ``Shot Orientation Controls for Interactive Cinematography with 360 video.'' \textit{UIST 2017} (22.5\% acceptance rate, 9 pages) & October 2017 \\
	\\

	\textbf{Amy Pavel}, Dan B Goldman, Björn Hartmann, Maneesh Agrawala. ``Vidcrit: Video-based Asynchronous Video Review.'' \textit{UIST 2016} (20.6\% acceptance rate, 12 pages) & October 2016 \\
	\\

	\textbf{Amy Pavel}, Dan B Goldman, Björn Hartmann, Maneesh Agrawala. ``SceneSkim: Searching and Browsing Movies Using Synchronized Captions, Scripts and Plot Summaries.'' \textit{UIST 2015} (23\% acceptance rate, 10 pages) & October 2015 \\
	\\

	Kurt Luther, Jari-lee Tolentino, Wei Wu, \textbf{Amy Pavel}, Brian P Bailey, Maneesh Agrawala, Björn Hartmann, Steven Dow. ``Structuring, Aggregating, and Evaluating Crowdsourced Design Critique.'' \textit{CSCW 2015} (28.3\% acceptance rate, 13 pages) & March 2015 \\
	\\

	\textbf{Amy Pavel}, Colorado Reed, Björn Hartmann, Maneesh Agrawala. ``Video Digests: A Browsable, Skimmable Format for Informational Lecture Videos.'' \textit{UIST 2014} (22.2\% acceptance rate, 10 pages) & October 2014 \\
	\\

\end{longtable}